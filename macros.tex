%%
%% macros.tex
%% 
%% Made by Alex Nelson <pqnelson@gmail.com>
%% Login   <alex@lisp>
%% 
%% Started on  2025-08-24T08:40:01-0700
%% Last update 2025-08-24T08:40:01-0700
%%
%% \def\topofcontents{\hsize 5.5in
%%   \vglue 0pt plus 1fil minus 1.5in
%%   \def\?##1]{\hbox to 1in{\hfil##1.\}}
%% }

\def\bul{\par\textindent{$\bullet$}}

\newif\ifpdf 
\ifx\pdfmajorversion\undefined \pdffalse \else \pdftrue \fi

\font\ninett=cmtt9
\font\eighttt=cmtt8
\font\nineit=cmti9
\font\eightit=cmti8

\input{epsf}
\input graphicx.tex %

\def\graphics#1{\ifpdf\includegraphics{#1.pdf}\else\epsfbox{#1.eps}\fi}

% Grammar related helpers
%\def\<#1>{\ifmmode\langle\hbox{\textit{#1\/}}\rangle\else$\langle$\hbox{\textit{#1\/}}$\rangle$\fi}
\def\<#1>{\hbox{$\langle$\textit{#1\/}$\rangle$}}
% for my unconfirmed suspicions, to make it easier to edit them out
% (Style based on Knuth's 1984 article on literate programming)
\def\Ithink#1{\ifhmode\ \fi$[\mkern-2mu[$#1$]\mkern-2mu]$\ } 
\def\hex{\hbox{$^{\scriptscriptstyle\#}$\tt\aftergroup}} % CWEB style
\def\H#1{\hbox{$^{\scriptscriptstyle\#}$\tt#1}}
% \def\H#1{} % hexadecimal \def\H#1{\hbox{\rm\char"7D\tt#1}} % hexadecimal constant
\def\shiftSix{\char`\^}
\def\pipe{\ifmmode\char`\|\else$\char`\|$\fi}
\def\botofcontents{\vskip 0pt plus 1fil minus 1.5in}
%\def\to{\;.\;.\;}
\def\dotdot{\mathrel{.\,.}} % double dot, used only in math mode

% figures, automatically number the figure
\newcount\fig
\fig=0
\def\figure{\global\advance\fig by 1\begingroup
\edef\modno{\the\fig}
}
\def\endfigure{\endgroup}
\def\caption#1{{\def\it{\eightit}\def\bf{\eightbf}\rightline{\vbox{\vskip-3pc\hbox{\eightrm{\eightbf Fig.~\the\fig.}
#1}}}}}

%%%
%%% XML schema
%%%
\def\schema{\obeylines\parindent=0pt\tt\frenchspacing}

% some common attributes
\def\pos#1{#1attribute col \LB\ xsd:integer \RB,

#1attribute line \LB\ xsd:integer \RB}
\def\idnr#1{#1attribute idnr \LB\ xsd:integer \RB}
\def\spelling#1{#1attribute spelling \LB\ text \RB}


%\def\to{\ifmmode\dotdot\else$\dotdot$\fi}

\let\oldOne\1
\newif\ifunindented\unindentedtrue
\def\interface{\ifunindented\def\1{\oldOne\gdef\1{}}\unindentedfalse\fi}
\def\endinterface{\unindentedtrue\global\let\1\oldOne\ignorespaces}

%%%
%%% Symbols
\def\section{\mathhexbox278} % TeX's \S
\def\pilcrow{\mathhexbox27B}
%%
\def\define#1{\textbf{``#1''}}

%% BEGIN twimac related hacks
\let\none\relax
\def\ET{ and~} % conjunction between two section numbers
\def\ETs{, and~} % conjunction between the last two of several section numbers
%% END twimac related hacks

\def\makeatletter{\catcode`\@11\relax}

\def\makeatother{\catcode`\@12\relax}
\makeatletter

%% Hacks to allow \pdfURL{} even when we're not making a PDF
\expandafter\ifx\csname pdfURL\endcsname\relax%
  \def\pdfURL#1#2{#2}
\fi

\def\@href#1#2{\leavevmode\pdfURL{#2}{#1}\catcode`\_8{}}

\def\href{\catcode`\_11\relax\@href}

\def\@doi#1{\leavevmode\pdfURL{{\tt doi:#1}}{https://doi.org/#1}\catcode`\_8\relax}

\def\doi{\catcode`\_11\relax\@doi}

\def\arXiv#1{\href{https://arxiv.org/abs/#1}{{\tt arXiv:#1}}}
%\def\doi#1{\href{https://doi.org/#1}{{\tt doi:#1}}}

% If not on PDF, just print the link
\expandafter\ifx\csname pdfnote\endcsname\relax%
  \def\pdfnote#1.{#1}
\fi

%%%
%%% A \label and \ref system tailored specifically for node numbers
%%%

% \label
\input{labels}

\newwrite\labelsfile
\immediate\openout\labelsfile={labels.tex}

\def\closelabels{\immediate\closeout\labelsfile%
  \gdef\closelabels{}}

\def\label#1{%
  \expandafter\def\csname l@#1\expandafter\endcsname\expandafter{%
    \modno%
  }%
  \immediate\write\labelsfile{%
    \string\expandafter\def\noexpand\csname l@#1\noexpand\endcsname{\modno}%
}\ignorespaces}

% \ref
\def\warnundefinedrefhandler#1{\message{Warning: reference #1 on page \folio undefined}{\bf??}}
\def\undefinedrefhandler#1{{\bf??}}
\let\defaultundefinedrefhandler\undefinedrefhandler

\def\ref#1{%
  \expandafter\ifx\csname l@#1\endcsname\relax%
    \warnundefinedrefhandler{#1}%
    \undefinedrefhandler{#1}%
  \else%
    \expandafter\expandafter\expandafter\relax\csname l@#1\endcsname%
  \fi}

% \xref{label} will create a linked text to the section with a \label
% if possible; otherwise, it will return bold "??"
\def\xref#1{%
  \edef\tmpref{\ref{#1}}%
  \expandafter\ifx\csname l@#1\endcsname\relax%
    \warnundefinedrefhandler{#1}%
    \undefinedrefhandler{#1}%
  \else%
    \expandafter\pdfnote\tmpref.%
  \fi}

% Redefine \bye to use \@@end, so we can redefine \end
\let\@@end\end

% `\enddocument` needs to have an \endgroup to fix
% "semi simple group (level 1) entered at line N (\begingroup)"
\def\bye{\par\vfill\supereject\closelabels\@@end}
\makeatother


%% Font names from LaTeX
\def\textbf#1{{\bf #1}}
\def\texttt#1{{\tt #1}}
\def\textit#1{{\it #1}}
\def\emph{\textit}

% When we want to add a label to a chunk without using @* (because we
% do not want to start a new page, or place it on the table of contents)
\def\node#1{\textbf{#1\enspace}\ignorespaces}


%%
%% notation.tex
%% 
%% Made by Alex Nelson <pqnelson@gmail.com>
%% Login   <alex@lisp>
%% 
%% Started on  2025-09-13T16:58:40-0700
%% Last update 2025-09-13T16:58:40-0700
%% 

% Notation macros
%%%
%%% LOGOS
%%%
\font\eightbf=cmbx8
\font\ninerm=cmr9
\let\mc=\ninerm % medium caps
\font\eightrm=cmr8
\let\sc=\eightrm % for smallish caps (NOT a caps-and-small-caps font)

\def\acro#1{{\mc #1\spacefactor1000}} % from tugboat

\def\CAUTION/{C{\sc AUTION}}
\def\ASCII/{{\mc ASCII\spacefactor1000}}
\def\UML/{{\mc UML\spacefactor1000}}
\def\WSM/{{\mc WSM\spacefactor1000}}
\def\UTF#1{{\mc UTF-#1\spacefactor1000}}

% Operating systems
\def\BeOS/{{\mc B}e\acro{OS}}
\def\BSD/{{\mc BSD\spacefactor1000}}
\def\GNU/{\acro{GNU}}
\def\macOS/{mac\acro{OS}} % tugboat.cmn
\def\MacOSX/{Mac\,\acro{OS\,X}} % tugboat.cmn
\def\POSIX/{{\mc POSIX\spacefactor1000}}
\def\QNX/{{\mc QNX\spacefactor1000}}
\def\SOLARIS/{S{\sc OLARIS\spacefactor1000}}
\def\UNIX/{{\mc U\kern-.05emNIX\spacefactor1000}}
\def\VAX{V\kern-.12em A\kern-.1em X\spacefactor1000} % tugboat.cmn

% Programming languages
\def\CEE/{{\mc C\spacefactor1000}} % cwebmac, the C programming language
\def\FORTRAN/{{\mc FORTRAN\spacefactor1000}}% taocpmac, ditto for the older guy
\def\ALGOL/{{\mc ALGOL\spacefactor1000}} % taocpmac
\def\COBOL/{{\mc COBOL\spacefactor1000}} % taocpmac
\def\MIX{{\tt MIX\spacefactor1000}} % taocpmac
\def\MMIX{{\tt MMIX\spacefactor1000}} % taocpmac
\def\PASCAL/{{\mc PASCAL\spacefactor1000}}
\def\TEX/{\TeX}
\def\CPLUSPLUS/{{\mc C\PP\spacefactor1000}} % cwebmac
\def\Cee{\CEE/} % for backward compatibility
\def\Cpp{\CPLUSPLUS/} % for backward compatibility
\newbox\PPbox % cwebmac, symbol for ++
\setbox\PPbox=\hbox{\kern.5pt\raise1pt\hbox{\sevenrm+\kern-1pt+}\kern.5pt} % cwebmac
\def\PP{\copy\PPbox} % cwebmac
% https://tex.stackexchange.com/q/44528/14751
% Also of note: https://tex.stackexchange.com/q/132846/14751
\def\CEESHARP/{{\mc C\#\spacefactor1000}}%{${\mc C}^{\sharp}$\spacefactor1000}
\def\CSHARP/{\CEESHARP/}

% From tugboat.sty
\def\ASCII/{\acro{ASCII}}%{\mc ASCII\spacefactor1000}}
\def\PDF/{\acro{PDF}}
\def\SQL/{{\mc SQL\spacefactor1000}}
\def\TANGLE/{{\tt TANGLE\spacefactor1000}}
\def\WEB/{{\tt WEB\spacefactor1000}}
\def\WEAVE/{{\tt WEAVE\spacefactor1000}}
\def\XML/{{\mc XML\spacefactor1000}}
\def\XSLT/{{\mc XSLT\spacefactor1000}}

%%% Miscellaneous macros
% \boxit from TeXbook exercise 21.3
% \setbox4=\vbox{\hsize 23pc\noindent\strut blah blah blah\strut}
% $$ \boxit{\box4} $$
%% \def\boxit#1{\leavevmode\hbox to0pt{\vbox to0pt{\hrule\hbox{\vrule\kern3pt%
%%   \vbox{\kern3pt#1\kern3pt}\kern3pt\vrule}\hrule}}}
\def\boxblock#1{\hbox{\vbox{\hrule\hbox{\vrule\kern3pt%
  \vbox{\kern3pt#1\kern3pt}\kern3pt\vrule}\hrule}}}
  
\def\boxit#1#2#3#4{\leavevmode
  \hbox{\vrule
    \vtop to#4{
      \vbox to#3{\hrule\kern#1%
        \hbox{\kern#1#2\kern#1}
      }%
      %\kern#1
      \hrule
    }%
  \vrule}%
}

%%%
%%% Mathematical notation
%%%

% notation
\def\abs#1{|#1|} % since |foo| means something else in WEB
\def\lcm{\mathop{\rm lcm}\nolimits}
\def\sgn{\mathop{\rm sgn}\nolimits}
\long\def\frac#1#2{{\begingroup #1\endgroup\over#2}}
\def\dom{\mathop{\rm dom}\nolimits}
\def\pto{\rightharpoonup} % partial function

% Logic
\def\implies{\;\Longrightarrow\;}
\def\iff{\;\Longleftrightarrow\;}

% Set theory
\let\propersubset\subset
\let\subset\subseteq

\def\CC{{\bf C}}
\def\NN{{\bf N}}
\def\QQ{{\bf Q}}
\def\ZZ{{\bf Z}}

\def\magnitude#1{\Vert #1\Vert}
\def\card#1{|#1|}


%%%
%%% List macros
%%%
\newcount\bugs\bugs=0
\def\startbugs{\smallbreak%
  \begingroup\advance\leftskip\parindent}%
\def\endbugs{\par\endgroup%
  \smallbreak}
\def\bug{\global\advance\bugs by1%
\par\hang\textindent{(\the\bugs)}}

\makeatletter
\let\@item\item

\newcount\enumi\enumi=1
\def\enumerate{\smallbreak%
  \begingroup%
  \advance\leftskip\parindent%
  \global\enumi=0%
  \def\item{\global\advance\enumi by1%
    \par\hang\textindent{(\the\enumi)}}%
  }
\def\endenumerate{\endgroup%
%\advance\leftskip-\parindent%\endgroup
\smallbreak}

\let\@item\item
\def\enumerate{\par%\smallbreak%
  \advance\leftskip\parindent%
  \global\enumi=0%
  \def\item{\global\advance\enumi by1%
    \@item{(\the\enumi)}}%
  }
\def\endenumerate{\par\global\let\item\@item%\endgroup%
\advance\leftskip-\parindent%\endgroup
\par}%\smallbreak}


\makeatother


%%%
%%% Algorithm ``environments''

\def\xskip{\hskip 7pt plus 3pt minus 4pt}

\newdimen\algindent
\newif\ifitempar \itempartrue % normally true unless briefly set false
\def\algindentset#1{\setbox0\hbox{{\bf #1.\kern.25em}}\algindent=\wd0\relax}
\def\algbegin #1 #2{\algindentset{#21}\alg #1 #2} % when steps all have 1 digit
\def\aalgbegin #1 #2{\algindentset{#211}\alg #1 #2} % when 10 or more steps
\def\alg#1(#2). {\medbreak % Usage: \algbegin Algorithm A (algname). This...
  \noindent{\bf#1}({\it#2\/}).\xskip\ignorespaces}
\def\algstep#1.{\ifitempar\smallskip\noindent\else\itempartrue
  \hskip-\parindent\fi
  \hbox to\algindent{\bf\hfil #1.\kern.25em}%
  \hangindent=\algindent\hangafter=1\ignorespaces}

\def\slug{\hbox{\kern1.5pt\vrule width2.5pt height6pt depth1.5pt\kern1.5pt}}

% indexing hack?
\def\9#1{}

%%%
%%% Table of contents hack
%%%

\newif\ifincontents\incontentsfalse
\newcount\chapno\chapno=0
\newcount\secno\secno=0
\newcount\subsecno\subsecno=0
\def\refstepchapter{\global\secno=0%
  \global\subsecno=0%
  \global\advance\chapno by1\message{INCREMENTING THE CHAPTER}}
\def\refstepsec{\global\subsecno=0%
  \global\advance\secno by1}
\def\refstepsubsec{\global\advance\subsecno by1}

\newif\ifinchapter\inchapterfalse
\newif\ifinsection\insectionfalse
\newif\ifinsubsection\insubsectionfalse

%\def\chapter#1{\global\inchaptertrue\ifincontents\llap{{\bf\the\chapno}\quad}\refstepchapter\fi\ignorespaces}

\def\chapter#1{\global\inchaptertrue\ifincontents\refstepchapter\llap{{\bf\the\chapno}\quad}\fi\ignorespaces}

%\def\chapter#1{\global\chaptertrue\xdef\thechapter{\the\chapno}\ifincontents\llap{{\bf\the\chapno}\quad}\refstepchapter\fi\ignorespaces}

\def\secn#1{\global\inchapterfalse%
%% \global\insectiontrue%
%% \global\insubsectionfalse%
\ifincontents\refstepsec\xdef\thesection{\the\secno}\hbox to2em{\thesection{.}\hss}\fi%
\ignorespaces}%\else$\vdash$\

\def\subsection#1{\global\inchapterfalse%
%% \global\insectionfalse%
%% \global\insubsectiontrue%
\ifincontents\refstepsubsec%
\xdef\thesubsection{\the\subsecno}%
\hbox to4em{\qquad\thesubsection{.}\hss}\fi%
\ignorespaces}%\else$\vdash\vdash$\ 

\def\initialchapno{0}

\def\setchapno#1{\global\chapno=#1\gdef\initialchapno{#1}}

\def\togglecontents{%
  %\gdef\chapter##1{\global\advance\chapno by1\llap{{\bf\the\chapno}\quad}\ignorespaces}%
  \global\incontentstrue%
  \global\chapno=\initialchapno%
  \global\secno=0%
  \global\subsecno=0}

\def\topofcontents{\global\incontentstrue\togglecontents%
  \centerline{\titlefont\title}\vskip.7in
  \vfill} % this material will start the table of contents page

%%%
%%% Hack for printing chapters and sections?
%%%
\font\tenssbx=cmssbx10
\font\twelvess=cmss12
\font\chaptertitlefont=cmssbx10 scaled\magstep2

% Borrowed from Knuth's "Art of Computer Programming"
\def\beginchapter#1: #2.{%
  %\vfill\eject
  \leftline{\twelvess #1} %\spaceskip=10pt \def\\{\kern1pt}
  \vskip 4pc
  \rightline{\chaptertitlefont #2}
  \vskip 2pc plus 1 pc minus 1 pc}

\def\beginsection#1: #2.{%
  \leftline{\tenssbx#1. \uppercase{#2}}
  \nobreak\smallskip\noindent}

\newdimen\spaceleft

% Ensure there's enough space for a subsection and the first numbered
% paragraph in the subsection. If not, just start on a new page.
\def\subsectionbreak{%
  \spaceleft=\vsize%
  \advance\spaceleft by-\pagetotal%
  \ifdim\spaceleft<2.5in%
    \vfill\eject%
  \else%
    \bigbreak%
  \fi%
}

\def\beginsubsection#1: #2.{%
  \subsectionbreak
  \leftline{\tenssbx#1. #2}
  \nobreak\smallskip\noindent}

\def\oldN#1.#2.{%
  \ifpdf \ifpdflua\relax
  \else \special{pdf: outline 0 << /Title (\the\toksE) /Dest
    [ @thispage /FitH @ypos ] >>}\fi\fi
  \ifon\startsection{\bf\ignorespaces#2.\quad}\ignorespaces}

\def\prefix{}

\newif\ifinfile\infilefalse


\def\newM#1.{\newMN#1.\ifon\vfil\penalty-100\vfilneg % beginning of section
  \global\inchapterfalse%
  \global\insectionfalse%
  \global\insubsectionfalse%
  \vskip\intersecskip\startsection\ignorespaces}

\def\N#1.  [#2] #3.{% beginning of starred section
  \if C#2\def\prefix{\chapter{}}\inchaptertrue%
  \else\if F#2\def\prefix{\chapter{}}\inchaptertrue\infiletrue%
  \else\if S#2\def\prefix{\secn{}}\insectiontrue%
  \else\if s#2\def\prefix{\subsection{}}\insubsectiontrue%
  \else\def\prefix{\noindent\ignorespaces}\fi\fi\fi\fi%
  \ifpdf{\makeoutlinetoks#3\outlinedone}\fi%
  \gtitle={#3}\MN#1.\ifinsubsection\else\vfill\eject\fi % define running headline
  \message{*\modno} % progress report
  \def\stripprefix##1>{}\expandafter\def\expandafter\gtitletoks\expandafter{\prefix #3}%
  \edef\gtitletoks{\expandafter\stripprefix\meaning\gtitletoks}%
  \ifinchapter%
    \refstepchapter%
    \ifinfile%
    \beginchapter{File \the\chapno}: #3.%
    \else%
    \beginchapter{Chapter \the\chapno}: #3.%
    \fi%
    \newM#1.%
  \else%
    \ifinsection%
      \refstepsec%
      \beginsection{Section \the\chapno.\the\secno}: #3.%
      \newM#1.%
    \else%
      \ifinsubsection\message{IN SUBSECTION}%
        \refstepsubsec%
        \beginsubsection{Subsection \the\chapno.\the\secno.\the\subsecno}: #3.%
        \newM#1.%
      \else% in all other cases
        \oldN#1.#3.%
      \fi%
    \fi%
  \fi%
  \global\infilefalse%
  \global\inchapterfalse%
  \global\insectionfalse%
  \global\insubsectionfalse%
  \edef\next{\write\cont{\Z{\gtitletoks}{\modno}% write to contents file
   {\noexpand\the\pageno}{\the\toksE}}}\next % \Z{title}{sec}{page}{ss}
\ignorespaces}

\def\M#1.{\MN#1.\ifon\vfil\penalty-100\vfilneg % beginning of section
  \global\inchapterfalse%
  \global\insectionfalse%
  \global\insubsectionfalse%
  \vskip\intersecskip\startsection\ignorespaces}

\def\MN#1.{\par % common code for \M, \N
  {\xdef\modstar{#1}\let\*=\empty\xdef\modno{#1}}% remove \* from section name
  \ifx\modno\modstar \onmaybe \else\ontrue \fi
  \mark{{{\tensy x}\modno}{\the\gtitle}}}
\def\newMN#1.{\par % common code for \M, \N
  {\xdef\modstar{#1}\let\*=\empty\xdef\modno{#1}}% remove \* from section name
  \ifx\modno\modstar \onmaybe \else\ontrue \fi
  \mark{{{\tensy x}\modno}{\the\gtitle}}}
\endinput

\newif\ifinx\inxfalse



%%%
%%% Failed attempt to make the header include the chapter (or
%%% section.subsection) number, and the footer include the page
%%% number.
%%%

\iffalse
\def\MN#1.{\par % common code for \M, \N
  {\xdef\modstar{#1}\let\*=\empty\xdef\modno{#1}}% remove \* from section name
  \ifx\modno\modstar \onmaybe \else\ontrue \fi
  \mark{{\pilcrow\modno}{\the\gtitle}}}
% each \mark is {section reference or null}{group title}

\def\lr{L}
\botmark{\the\pageno}
%\newtoks\footline
\footline={\ifodd\pageno%
  \if L\lr\hskip\pagewidth\fi\hss\the\pageno%\fi%
\else%
  \the\pageno\hss%
\fi}


\def\thesection{\ifnum\secno>0\section\the\secno\fi}%\ifnum\subsecno>0.\the\subsecno\fi\fi}
\def\thesubsection{\thesection\ifnum\subsecno>0{.\the\subsecno}\fi}
\def\lheader{\mainfont\uppercase\expandafter{\romannumeral\chapno}\eightrm\qquad\grouptitle
  \hfill\title\qquad\mainfont\topsecno} % top line on left-hand pages
\def\rheader{\mainfont\topsecno\eightrm\qquad\title\hfill
  \grouptitle\qquad\mainfont\thesubsection} % top line on right-hand pages

%\advance\hsize by24pt
%\advance\fullpageheight by24pt

\def\makefootline{\baselineskip=24pt \lineskiplimit=0pt%
  \ifodd\pageno\rightline{\the\footline}%
  \else\leftline{\the\footline}\fi}

% Chapter 23 of TeXbook for output routines
%\let\page=\pagebody \raggedbottom % cweb 3.2 uses this instead of next line
\def\page{\box255 }\normalbottom % faster, but loses plain TeX footnotes
\def\normaloutput#1#2#3{\ifodd\pageno\hoffset=\pageshift\fi
  \shipout\vbox{
    \vbox to\fullpageheight{
      \iftitle\global\titlefalse
      \else\hbox to\pagewidth{\vbox to10pt{}\ifodd\pageno #3\else#2\fi}\medbreak\fi%
      \vfill#1%
      \medbreak\hbox to\pagewidth{\ifinx\ifodd\pageno\hss\fi\fi\makefootline}%
      }} % parameter #1 is the page itself
  \global\advance\pageno by1}

\def\inx{\global\inxtrue\par\vskip6pt plus 1fil % we are beginning the index
  \def\page{\box255 } \normalbottom
  \write\cont{} % ensure that the contents file isn't empty
       \write\cont{\catcode `\noexpand\@=12\relax}   % \makeatother
  \closeout\cont % the contents information has been fully gathered
  \output{\ifpagesaved\normaloutput{\box\sbox}\lheader\rheader\fi
    \global\setbox\sbox=\page \global\pagesavedtrue \mark{\topmark}}
  \pagesavedfalse \eject % eject the page-so-far and predecessors
  \setbox\sbox\vbox{\unvbox\sbox} % take it out of its box
  \vsize=\pageheight \advance\vsize by -\ht\sbox % the remaining height
  \hsize=.5\pagewidth \advance\hsize by -10pt
    % column width for the index (20pt between cols)
  \ifhint\else
  \parfillskip 0pt plus .6\hsize % try to avoid almost empty lines
  \fi
  \def\lr{L} % this tells whether the left or right column is next
  \output{\if L\lr\global\setbox\lbox=\page \gdef\lr{R}
    \else\normaloutput{\vbox to\pageheight{\box\sbox\vss
        \hbox to\pagewidth{\box\lbox\hfil\page}}}\lheader\rheader
    \global\vsize\pageheight\gdef\lr{L}\global\pagesavedfalse\fi}
  \message{Index:}
  \parskip 0pt plus .5pt
  \outer\def\:##1, ##2.{\par\hangindent2em\noindent##1:\kern1em
    \scan##2!.} % index entry
  \let\ttentry=\. \def\.##1{\ttentry{##1\kern.2em}} % give \tt a little room
  \def\[##1]{$\underline{\scan##1!}$\scan} % underlined index item
  \ifacrohint\def\digits{\pdflink{\the\countA}\scan}
  \else\def\digits{{\the\countA}\scan}\fi
  \def\scan##1{\begingroup
    \ifx!##1% exit on exclamation point
    \else\ifx,##1,\space\aftergroup\scan % insert ,\space}\scan...
    \else\ifx\[##1\aftergroup##1% insert }\[...
    \else\ifx\*##1\aftergroup\lapstar\aftergroup\scan % insert }\lapstar\scan...
    \else\ifnum`##1>`9##1\aftergroup\scan % insert #1}\scan...
    \else\ifnum`##1<`0##1\aftergroup\scan % insert #1}\scan...
    \else
      \afterassignment\digits \aftergroup\countA
      \aftergroup##1% insert }\countA=#1\digits...
    \fi\fi\fi\fi\fi \fi
    \endgroup}
  \rm \rightskip0pt plus 2.5em \tolerance 10000
  \hyphenpenalty 10000 \parindent0pt}
\fi



\endinput